%include local.fmt

\def\prod{\mathrel{::=}}
\def\altn{\mathrel{\mid}}
\def\NT#1{\mathsf{#1}}
\def\Nat{\mathbb{N}}
\def\Expression{\NT{Expression}}

\chapter{Semantics for Compiler Correctness}

%\begin{itemize}
%\item denotational, small-step for $(+,N)$
%\item equivalence proof
%\item rule induction (coinduction?)
%\item machine semantics for $\{PUSH m, ADD\} x N$
%\item statement of compiler correctness (style? direct vs CPS)
%\item compiler correctness as running example: proof
%\end{itemize}

In the context of computer science, semantics is the study of the meaning of
programming languages. Having a mathematically rigorous definition of the
language allows us to reason about (programs written in) the language
precisely and without ambiguity. In this chapter, we will take an elementary
look at different ways of giving meaning to a language, and various
techniques for proving properties about programs. After this prelude, we
reach the nub of this chapter, where we shall define a simple compiler from
an expression language to a stack machine, and explore what it means to say
that the compiler is correct.

\section{Semantics}%{{{%

\subsection{Numbers and Addition}%{{{%

To unambiguously reason about what any given program means, we need to give
a mathematically rigorous definition of the language in which it is
expressed. To this end, let us consider the elementary language of natural
numbers and
addition~\cite{hutton04-exceptions,hutton06-calculating,hutton07-interruptions}.
\begin{gather*}
	\Expression \prod \Nat \altn \Expression \oplus \Expression
\end{gather*}
That is, an $\Expression$ is either simply a natural number, or a pair of
$\Expression$s, punctuated with the $\oplus$ symbol to represent the
operation of addition. We will adhere to a naming convention of $m, n \in
\Nat$ and $a, b, e \in \Expression$.

Although seemingly simplistic, this language has sufficient structure to
illustrate the essential aspects of computation, namely that of sequencing
computations and combining their results, as we shall expand upon later in
section \ref{sec:small-step}.

%}}}%

\subsection{Denotational Semantics}%{{{%

% semantic brackets
\def\sb[#1]{[\![#1]\!]}

Denotation semantics attempts to give an interpretation of the source
language in some suitable existing formalism that we already understand.
More specifically, the denotation of a program is a representation of what
the program means in the vocabulary of the chosen formalism, which could be
the language of sets and functions, the $\lambda$-calculus, or perhaps one
of the many process calculi. Thus, to formally give a denotational semantics
for a language is to define a mapping from the source language into some
underlying semantic domain.

For example, we can give the following semantics for our earlier
$\Expression$ language, where each term is denoted by a natural number:
\begin{align*}
	\sb[\anonymous] &: \Expression \rightarrow \Nat \\
	\sb[ m ] &= m \\
	\sb[ a \oplus b ] &= \sb[ a ] + \sb[ b ]
\end{align*}
Here, a numeric $\Expression$ is interpreted as just the number itself. The
denotation of the $\oplus$ operator is $+$ on natural numbers;
alternatively, we could say that $a \oplus b$ is denoted by the sum of the
denotations of its subexpressions $a$ and $b$.

%}}}%

\subsection{Big-Step Semantics}%{{{%

\def\ruleName#1{\textsf{#1}}
\def\ruleTag#1{\quad(\ruleName{#1})}

The related notion of big-step semantics is concerned with the overall
result of a computation. Formally, we define a relation ${\Downarrow}
: \Expression \times \Nat$ between $\Expression$s and their final values,
given below in a natural deduction style:
\begin{gather*}
\frac{\rule{0pt}{1pt}}%
	{m \Downarrow m} \ruleTag{$\Downarrow$-val} \\[2ex]
\frac{a \Downarrow m \quad b \Downarrow n}%
	{a \oplus b \Downarrow m + n} \ruleTag{$\Downarrow$-add}
\end{gather*}
The first \ruleName{$\Downarrow$-val} rule says that a simple numeric
$\Expression$ evaluates to the number itself. The second
\ruleName{$\Downarrow$-add} rule states that, if $a$ evaluates to $m$ and
$b$ evaluates to $n$, then $a \oplus b$ evaluates to the sum $m + n$.

One intrinsic advantage of a big-step semantics is that the semantics can be
non-deterministic, in the sense that each expression could potentially
evaluate to multiple distinct values. In contrast, a denotational semantics
deals with non-determinism in the source language by mapping it to
a potentially different notion of non-determinism in the underlying
formalism---the semantics of the previous section must instead denote each
expression as a set of possible numbers.

%}}}%

\subsection{Small-Step Semantics}\label{sec:small-step}%{{{%

Small-step semantics on the other hand is concerned with how a computation
proceeds step-by-step. Again, we formally define a relation ${\mapsto}
: \Expression \times \Expression$, between pairs of $\Expression$s in this
instance:
\begin{gather*}
\frac{}%
	{m \oplus n \mapsto m + n} \ruleTag{$\mapsto$-add} \\[2ex]
\frac{b \mapsto b'}%
	{m \oplus b \mapsto m \oplus b'} \ruleTag{$\mapsto$-right} \\[2ex]
\frac{a \mapsto a'}%
	{a \oplus b \mapsto a' \oplus b} \ruleTag{$\mapsto$-left}
\end{gather*}
The first rule \ruleName{$\mapsto$-add} deals with the case where the
expressions on both sides of $\oplus$ are numerals: in a single step, it
reduces to the sum $m + n$. The second \ruleName{$\mapsto$-right} rule
applies when the left argument of $\oplus$ is a numeral, in which case the
right argument makes a single reduction, while \ruleName{$\mapsto$-left}
reduces the left argument of $\oplus$ whenever possible. No rule corresponds
to a lone numeric $\Expression$ as no further reductions are possible in
this case.

While the above rules give a left-to-right reduction order for
$\Expression$s, it would be perfectly reasonable to give the converse
right-to-left definition, or even to have a non-deterministic interleaved
reduction order.

\TODO{Blurb on monoids, and correspondence with monads.}

%}}}%

%}}}%

\section{Equivalence Proof and Techniques}%{{}}%

\section{Stack Machines and Their Semantics}%{{}}%

\section{Compiler Correctness}%{{}}%

% vim: ft=tex:

