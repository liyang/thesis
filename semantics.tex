%include local.fmt

\def\prod{\mathrel{::=}}
\def\altn{\mathrel{\mid}}
\def\NT#1{\mathsf{#1}}
\def\Nat{\mathbb{N}}
\def\Expression{\NT{Expression}}

\chapter{Semantics for Compiler Correctness}

%\begin{itemize}
%\item denotational, small-step for $(+,N)$
%\item equivalence proof
%\item rule induction (coinduction?)
%\item machine semantics for $\{PUSH m, ADD\} x N$
%\item statement of compiler correctness (style? direct vs CPS)
%\item compiler correctness as running example: proof
%\end{itemize}

In the context of computer science, the semantics of a programming language
is what we mean it to mean. Having a mathematically rigorous definition of
the language allows us to reason about (programs written in) the language
precisely and without ambiguity. In this chapter, we will take an elementary
look at different ways of giving meaning to a language, and various
techniques for proving properties about programs. After this prelude, we
reach the nub of this chapter, where we shall define a simple compiler from
an expression language to a stack machine, and explore what it means to say
that the compiler is correct.


\section{Numbers and Addition}%{{{%

To unambiguously reason about what any given program means, we need to give
a mathematically rigorous definition of the language in which it is
expressed. To this end, let us consider the elementary language of natural
numbers and
addition~\cite{hutton04-exceptions,hutton06-calculating,hutton07-interruptions}.
\begin{gather*}
	\Expression \prod \Nat \altn \Expression \oplus \Expression
\end{gather*}
That is, an $\Expression$ is either simply a natural number, or a pair of
$\Expression$s, punctuated with the $\oplus$ symbol to represent the
operation of addition.

Although seemingly simplistic, this language has sufficient structure to
illustrate an essential aspect of computation, namely that of sequencing, as
we shall see later in section \ref{sec:small-step}.

%}}}%

\section{Denotational and Big-Step Semantics}%{{{%

The denotational style takes a top-down, compositional view of semantics,
and assigns meaning to a given program in terms of the semantics of its
constituent parts. 

Thus a denotation for a language is formally a mapping from the source
language into some underlying semantic domain. 

\def\sb[#1]{[\![#1]\!]}

\begin{align*}
	\sb[ m ] &= m \\
	\sb[ a \oplus b ] &= \sb[ a ] + \sb[ b ]
\end{align*}





Closely related is the notion of big-step semantics, where 

Functions are more straightforward, but does not generalise to
non-deterministic semantics.

As a relation:

%}}}%

\section{Small-Step Semantics}\label{sec:small-step}%{{}}%

\section{Equivalence Proof and Techniques}%{{}}%

\section{Stack Machines and Their Semantics}%{{}}%

\section{Compiler Correctness}%{{}}%

% vim: ft=tex:

