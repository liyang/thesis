\documentclass[12pt,twoside,openright]{book}

\usepackage{setspace}
\doublespacing
\usepackage[left=1.5in,right=1in,top=1in,bottom=1in]{geometry}

% display date and time of when PDF was made
%\input{now}
\usepackage{fancyhdr}
\pagestyle{fancy}
\fancyhead{}
\fancyhead[LE]{\leftmark}
\fancyhead[RO]{\rightmark}
\fancyfoot{}
\fancyfoot[LE,RO]{\thepage}
%\fancyfoot[RE,LO]{\scriptsize\rightnow}
\renewcommand{\headrulewidth}{0pt}
\headheight 14.5pt


\usepackage{url}
\usepackage{longtable}
\usepackage{multirow}
\usepackage{comment}
\usepackage{marvosym}

\input{polycode.lhs.tex}

%\def\TODO#1{\par\noindent{}TODO:~\ldots\emph{#1}\ldots\par}
\def\TODO#1{\noindent{}TODO:~\ldots\emph{#1}\ldots}

\def\source#1{}

\usepackage{amsmath}
\usepackage{amsthm}
\newtheorem{theorem}{Theorem}[chapter]
\newtheorem{lemma}[theorem]{Lemma}
\newtheorem{corollary}[theorem]{Corollary}

\def\eqName#1{\textsf{\mdseries(#1)}}
\def\eqTag#1{\tag*{\eqName{#1}}}


\usepackage{mathpartir}
\def\TirName#1{\eqName{#1}}

\usepackage[pdftex,matrix,arrow,curve,color]{xy}
% for arrow labels in xymatrix
\usepackage{rotating}
\def\xyrot#1#2{\begin{turn}{#1}\parbox[c][1.5ex][c]{1.5ex}{\makebox[1.5ex][c]{$#2$}}\end{turn}}
% agda-mode doesn't like {-
\def\xyar{\ar@{->}}
\def\xyarr{\ar@{->>}}

% vector lookup for verified chapter
\def\index[#1]{\!\texttt[#1\texttt]\!}

% lhs2TeX
%{{{%
\usepackage[pdftex]{color}

\renewcommand\hsindent[1]{\quad}
\setlength{\mathindent}{2ex}
\DeclareMathAlphabet{\mathkw}{OT1}{cmss}{bx}{n}

\newcommand\keyw[1]{\textcolor[rgb]{.75,.4,0}{\mathkw{#1}}}
\newcommand\type[1]{\textcolor[rgb]{0,0,.75}{\mathsf{#1}}}
\newxyColor{type}{0 0 0.75}{rgb}{}
% \newcommand\tcls[1]{\textcolor[rgb]{.25,0,.75}{\mathsf{#1}}}
\newcommand\cons[1]{\textcolor[rgb]{0,.5,0}{\mathsf{#1}}}
\newcommand\dstr[1]{\textcolor[rgb]{.75,.75,0}{\mathsf{#1}}}
\newcommand\func[1]{\textcolor[rgb]{.75,0,0}{\mathsf{#1}}}
\newcommand\ltrl[1]{\textcolor[rgb]{.6,0,.6}{\mathtt{#1}}}
\newcommand\name[1]{\textcolor[rgb]{.75,0,.75}{\mathsf{#1}}}
\newcommand\commentstyle[1]{\textcolor[rgb]{0,.6,0.75}{#1}}
\newcommand\shed[1]{\colorbox[rgb]{.6,1,.6}{#1}}

\def\hide#1{}

\newcommand\Prime{\ensuremath{'}}
\newcommand\PPrime{\ensuremath{''}}
\newcommand\PPPrime{\ensuremath{'''}}

\newcommand\prefix[1]{#1\,}
\newcommand\postfix[1]{\,#1}

\newcommand\infix[1]{\mathbin{#1}}
\newcommand\infixL[1]{\infix{#1}\!}
\newcommand\infixM[1]{\!\infix{#1}\!}
\newcommand\infixR[1]{\!\infix{#1}}
%}}}%



% \includeonly{semantics,agda.lagda}
\begin{document}

\pagestyle{empty}
\begin{titlepage}
\begin{center}
\vspace*{1in}
{\LARGE Compiling Concurrency Correctly}
\par
{\Large Verifying Software Transactional Memory}
\par
\vspace{1.5in}
{\large Liyang HU}
\par
\vfill
A Thesis submitted for the degree of Doctor of Philosophy
\par
\vspace{0.5in}
School of Computer Science
\par
\vspace{0.5in}
University of Nottingham
\par
\vspace{0.5in}
June 2012
\end{center}
\cleardoublepage
\end{titlepage}


\pagestyle{fancy}
\pagenumbering{roman}

%{{{%
\begin{center}
{\large\bf Abstract}
\end{center}

Concurrent programming is notoriously difficult, but with multi-core
processors becoming the norm, is now a reality that every programmer must
face. Concurrency has traditionally been managed using low-level mutual
exclusion \emph{locks}, which are error-prone and do not naturally support
the compositional style of programming that is becoming indispensable for
today's large-scale software projects.

A novel, high-level approach that has emerged in recent years is that of
\emph{software transactional memory} (STM), which avoids the need for
explicit locking, instead presenting the programmer with a declarative
approach to concurrency. However, its implementation is much more complex
and subtle, and ensuring its correctness places significant demands on the
compiler writer.

This thesis considers the problem of formally verifying a compiler for STM.
Utilising a minimal language incorporating only the features that we are
interested in studying, we first explore various STM design choices, along
with the issue of compiler correctness via the use of automated testing
tools. Then we outline a new approach to concurrent compiler correctness
using the notion of bisimulation, implemented using the Agda theorem prover.
Finally, we show how bisimulation can be used to establish the correctness
of a low-level implementation of software transactional memory.

%}}}%

\chapter*{Acknowledgements and Thanks}

\ldots

%\begin{itemize}
%\item GMH for his guidance, patience and keeping me on track
%\item CTM for infecting me with DTP
%\item PGH for discussions on transactions
%\item NAD for Agda assistance and stdlib
%\item Ulf for developing Agda
%\item RKSD for figuratively\footnote{xkcd-figuratively} beating me into
%submission
%\item AJ for being my muse and therapist, particularly during the more
%emotionally difficult parts of the write-up
%\item TD's seemingly inexhaustible enthusiasm for inspiring me to get the
%hell on with it
%\item something about my parents (not forcing me to get a real job?)
%\end{itemize}

\tableofcontents
%\listoffigures

%{\include{introduction.lhs}}
%{\include{stm.lhs}}
%{%include local.fmt

\def\prod{\mathrel{::=}}
\def\altn{\mathrel{\mid}}
\def\NT#1{\mathsf{#1}}
\def\Nat{\mathbb{N}}
\def\Expression{\NT{Expression}}

\chapter{Semantics for Compiler Correctness}

%\begin{itemize}
%\item denotational, small-step for $(+,N)$
%\item equivalence proof
%\item rule induction (coinduction?)
%\item machine semantics for $\{PUSH m, ADD\} x N$
%\item statement of compiler correctness (style? direct vs CPS)
%\item compiler correctness as running example: proof
%\end{itemize}

In the context of computer science, semantics is the study of the meaning of
programming languages. Having a mathematically rigorous definition of the
language allows us to reason about (programs written in) the language
precisely and without ambiguity. In this chapter, we will take an elementary
look at different ways of giving meaning to a language, and various
techniques for proving properties about programs. After this prelude, we
reach the nub of this chapter, where we shall define a simple compiler from
an expression language to a stack machine, and explore what it means to say
that the compiler is correct.

\section{Semantics}%{{{%

\subsection{Numbers and Addition}%{{{%

To unambiguously reason about what any given program means, we need to give
a mathematically rigorous definition of the language in which it is
expressed. To this end, let us consider the elementary language of natural
numbers and
addition~\cite{hutton04-exceptions,hutton06-calculating,hutton07-interruptions}.
\begin{gather*}
	\Expression \prod \Nat \altn \Expression \oplus \Expression
\end{gather*}
That is, an $\Expression$ is either simply a natural number, or a pair of
$\Expression$s, punctuated with the $\oplus$ symbol to represent the
operation of addition. We will adhere to a naming convention of $m, n \in
\Nat$ and $a, b, e \in \Expression$.

Although seemingly simplistic, this language has sufficient structure to
illustrate the essential aspects of computation, namely that of sequencing
computations and combining their results, as we shall expand upon later in
section \ref{sec:small-step}.

%}}}%

\subsection{Denotational Semantics}%{{{%

% semantic brackets
\def\sb[#1]{[\![#1]\!]}

Denotation semantics attempts to give an interpretation of the source
language in some suitable existing formalism that we already understand.
More specifically, the denotation of a program is a representation of what
the program means in the vocabulary of the chosen formalism, which could be
the language of sets and functions, the $\lambda$-calculus, or perhaps one
of the many process calculi. Thus, to formally give a denotational semantics
for a language is to define a mapping from the source language into some
underlying semantic domain.

For example, we can give the following semantics for our earlier
$\Expression$ language, where each term is denoted by a natural number:
\begin{align*}
	\sb[\anonymous] &: \Expression \rightarrow \Nat \\
	\sb[ m ] &= m \\
	\sb[ a \oplus b ] &= \sb[ a ] + \sb[ b ]
\end{align*}
Here, a numeric $\Expression$ is interpreted as just the number itself. The
denotation of the $\oplus$ operator is $+$ on natural numbers;
alternatively, we could say that $a \oplus b$ is denoted by the sum of the
denotations of its subexpressions $a$ and $b$.

%}}}%

\subsection{Big-Step Semantics}%{{{%

\def\ruleName#1{\textsf{#1}}
\def\ruleTag#1{\quad(\ruleName{#1})}

The related notion of big-step semantics is concerned with the overall
result of a computation. Formally, we define a relation ${\Downarrow}
: \Expression \times \Nat$ between $\Expression$s and their final values,
given below in a natural deduction style:
\begin{gather*}
\frac{\rule{0pt}{1pt}}%
	{m \Downarrow m} \ruleTag{$\Downarrow$-val} \\[2ex]
\frac{a \Downarrow m \quad b \Downarrow n}%
	{a \oplus b \Downarrow m + n} \ruleTag{$\Downarrow$-add}
\end{gather*}
The first \ruleName{$\Downarrow$-val} rule says that a simple numeric
$\Expression$ evaluates to the number itself. The second
\ruleName{$\Downarrow$-add} rule states that, if $a$ evaluates to $m$ and
$b$ evaluates to $n$, then $a \oplus b$ evaluates to the sum $m + n$.

One intrinsic advantage of a big-step semantics is that the semantics can be
non-deterministic, in the sense that each expression could potentially
evaluate to multiple distinct values. In contrast, a denotational semantics
deals with non-determinism in the source language by mapping it to
a potentially different notion of non-determinism in the underlying
formalism---the semantics of the previous section must instead denote each
expression as a set of possible numbers.

%}}}%

\subsection{Small-Step Semantics}\label{sec:small-step}%{{{%

Small-step semantics on the other hand is concerned with how a computation
proceeds step-by-step. Again, we formally define a relation ${\mapsto}
: \Expression \times \Expression$, between pairs of $\Expression$s in this
instance:
\begin{gather*}
\frac{}%
	{m \oplus n \mapsto m + n} \ruleTag{$\mapsto$-add} \\[2ex]
\frac{b \mapsto b'}%
	{m \oplus b \mapsto m \oplus b'} \ruleTag{$\mapsto$-right} \\[2ex]
\frac{a \mapsto a'}%
	{a \oplus b \mapsto a' \oplus b} \ruleTag{$\mapsto$-left}
\end{gather*}
The first rule \ruleName{$\mapsto$-add} deals with the case where the
expressions on both sides of $\oplus$ are numerals: in a single step, it
reduces to the sum $m + n$. The second \ruleName{$\mapsto$-right} rule
applies when the left argument of $\oplus$ is a numeral, in which case the
right argument makes a single reduction, while \ruleName{$\mapsto$-left}
reduces the left argument of $\oplus$ whenever possible. No rule corresponds
to a lone numeric $\Expression$ as no further reductions are possible in
this case.

While the above rules give a left-to-right reduction order for
$\Expression$s, it would be perfectly reasonable to give the converse
right-to-left definition, or even to have a non-deterministic interleaved
reduction order.

\TODO{Blurb on monoids, and correspondence with monads.}

%}}}%

%}}}%

\section{Equivalence Proof and Techniques}%{{}}%

\section{Stack Machines and Their Semantics}%{{}}%

\section{Compiler Correctness}%{{}}%

% vim: ft=tex:

}
%{\include{testing.lhs}}
%{\include{model.lhs}}
%{\include{agda.lagda}}
%{\include{nondet.lagda}}
%{\include{fork.lagda}}
%%{\include{verified.lagda}}
{\include{atomic.lagda}}
%{\chapter{Conclusion}

To conclude, these final pages will comprise an overview of this thesis and
an account of how it came to be, followed by some directions for further
work.

\section{Retrospection}%{{{%

The quest for higher-level abstractions to manage the complexities of
concurrent programming has been an especially apt topic in recent years, due
to reasons outlined in the introductory chapter. With respect to software
transactional memory (Chapter \ref{ch:stm}), I was fortunate enough to be in
the right places at the right times to have attended two of Tim Harris's
talks on the topic: the first in Cambridge during my undergraduate years, on
the JVM-based implementation; and a second time at Imperial College during
my MSc course in the early part of 2005, on the composability of STM
Haskell.

My work for this thesis began in 2006 under the guidance of Graham Hutton,
with an initial goal of reasoning about concurrent programs, in particular
those written using STM. To this end, we opted for a simple formal language
based on Hutton's Razor, extended with a minimal set of transactional
primitives, described in Chapter \S\ref{ch:model}. While this
language---following the reference stop-the-world semantics given by Harris
et al.~\cite{harris05-composable}---had a simple implementation, it was not
immediately clear how STM Haskell dealt with conflicting transactions
internally, consequently drawing our attention towards the correctness of
the low-level concurrent implementation.

To better understand the implementation issues behind software transactional
memory, we began building a stack-based virtual machine and a compiler for
our minimal language, of which the final version is given in
\S\ref{sec:model-machine}. Using Haskell as a metalanguage, it was
a straightforward task to transcribe the syntax and semantics of our model
as an executable program. Combined with the use of QuickCheck and the
Haskell Program Coverage toolkit (Chapter \ref{ch:qc+hpc}), this allowed us
to take a `rapid prototyping' approach to the design of the virtual machine.
Most notably, we were able to clarify the appropriate conditions needed for
a transaction to commit successfully (\S\ref{sec:model-equality}), and to
realise that writing to an as-yet unread variable within a transaction does
not imply a dependency on the current state of the heap.

Of course, regardless of how many times we run QuickCheck on our STM model,
overwhelming evidence does not constitute a proof of compiler correctness
for our interleaved semantics and log-based implementation of transactional
memory. Due to the influence of numerous type-theorists at Nottingham, I had
become interested in dependently-typed programming (Chapter \ref{ch:agda})
and dually, the application of intuitionistic type theory as a framework for
conducting formal proofs. Therefore, the goal of a mechanised compiler
correctness proof for our model in a dependently-typed
language/proof-assistant seemed a natural choice.

Whereas compiler correctness theorems in a deterministic setting (Chapter
\ref{ch:semantics}) are concerned only with the final results, with the
introduction of non-determinism (of which concurrency is one form) we can no
longer afford to ignore \emph{how} results are computed, in addition to what
is being computed. Bisimilarity as a notion of behavioural equivalence is
a standard tool in the field of process calculi, and
Wand~\cite{wand95-parallel,wand95-denotational,gladstein96-concurrent} et
al.~were the first to use it to tackle concurrent compiler correctness over
a decade ago. Their work relied on giving denotational semantics to both
source and target languages in an underlying process calculus, and showing
that compilation preserved bisimilarity of denotations. In contrast, we
defined our language and virtual machine in an operational manner, and
sought a simpler and more direct approach.

Thus, the idea of the combined machine was born, detailed in Chapter
\ref{ch:nondet}. A key realisation was that certain kinds transitions
preserve bisimilarity, giving rise to the $\func{elide\text-\tau}$ lemma.
Having tested the waters with the non-deterministic Zap language, Chapter
\ref{ch:fork} then demonstrates that our approach can indeed scale to handle
concurrency, at least for that of the Fork language. As well as updating the
$\func{elide\text-\tau}$ lemma for explicit concurrency, we also showed that
combining bisimilar pairs of thread soups preserves bisimilarity.

Finally, Chapter \ref{ch:verified} brings our object language up to par with
our earlier STM model by introducing an $\cons{atomic}$ construct that
follows a stop-the-world semantics, in contrast with its virtual machine's
log-based implementation that allows concurrent transactions to interleave
each other. The final compiler correctness proof makes essential use of
a notion of equivalence between the heap of the high-level semantics and
that of the virtual machine, overlaid with read and write logs.

%}}}%

\section{Summary of Contributions}%{{{%

This thesis addresses the familiar question of compiler correctness in
a current context, with a particular emphasis on the implementation of
software transactional memory. We have identified a simplified subset of STM
Haskell that is amenable to formal reasoning, which has a stop-the-world
transactional semantics, together with a concurrent virtual machine for this
language, using the notion of transaction logs. A compiler linking this
simplified language to its virtual machine then allowed us to formulate
a concrete statement of compiler correctness. We were able to implement the
above semantics in a fairly direct manner using the high-level vocabulary
provided by Haskell, enabling us to empirically test compiler correctness
with the help of QuickCheck and HPC.

Working towards a formal proof of the above hypothesis, we stripped down to
a minimal language with trivial non-determinism, and moving to a labelled
transition system. The core idea of a combined machine and semantics then
allowed us to establish a direct bisimulation between this language and its
virtual machine. This technique was put into practice using the Agda proof
assistant, giving a machine-checked compiler correctness proof for
a language with a simple notion of non-determinism. We then extended the
above proof and our approach in the direction of the initially identified
subset of STM Haskell, in an incremental manner: first introducing explicit
concurrency in the form of a $\cons{fork}$ primitive, before finally
extending this language with an $\cons{atomic}$ construct. We believe this
to be the first machine-checked compiler correctness proof showing the
equivalence of the log-based approach and the stop-the-world semantics for
implementing transactions.

%}}}%

\section{Directions for Further Research}%{{{%

Our simplified model of STM Haskell focuses on the essence of implementing
transactions, and consequently omits many of the facilities expected of
a realistic language. Namely, the lack of primitive recursion or even name
binding limits the computational power of our model in a very tangible
sense. We could be tackle this using a lightweight approach, by borrowing
said facilities from the metalanguage and defining the high-level semantics
as a functional specification. For example, Gordon's
thesis~\cite{gordon92-fpio} presents such a specification of teletype IO for
a subset of Haskell in terms of a low-level metalanguage, while
Swierstra~\cite{swierstra08-funspec} advocates the use of Agda as the
metalanguage, due to its enforcement of totality.

Given the above as a basis, a machine-verified formalisation of the omitted
parts of the STM Haskell specification---in particular
$\func{retry}$/$\func{orElse}$ and the interaction with exceptions---becomes
a much more tractable proposition. Open questions include: How will these
additions affect the design of the corresponding virtual machine? Can we
maintain the simplicity of our combined machine approach? Is the outline of
our reasoning for transactions still valid in this richer language? Our
current virtual machine immediately retries failed transactions, rather than
waiting until some relevant transactional variable has changed. How can our
virtual machine more faithfully model the implementation of STM in GHC?

Going further, we could extend the set of side-effects that can be safely
rolled back by the transactional machinery. One widely asserted advantage of
STM Haskell over other STM implementations is that its type system restricts
transactions---aside from modifying $\type{TVar}$s---to pure computations,
guaranteeing that rollback is always possible. During my initial work on the
model of STM Haskell, the notion of running multiple nested transactions
concurrently arose quite naturally, when considering their r\^{o}le in the
implementation of $\func{retry}$/$\func{orElse}$. While the $\func{forkIO}$
primitive is considered impure, forking a nested transaction need not be, as
its side-effects can only escape as part of that of its enclosing
transaction. It could be interesting to flesh out the precise behaviour of
such
a $\func{forkSTM}\;\keyw{::}\;\type{STM\;()\;\rightarrow\;STM\;ThreadId}$
primitive, and to evaluate its utility for concurrent programming in the
real world.

%}}}%

%\section*{Closing Remarks}

%Thanks for making it this far. It felt like I almost didn't.

% vim: ft=tex:

}

\bibliographystyle{alpha}
\bibliography{thesis}

\end{document}

