\documentclass[12pt,twoside]{report}

\usepackage{setspace}
\doublespacing
\usepackage[left=1.5in,right=1in,top=1in,bottom=1in]{geometry}

% display date and time of when PDF was made
\usepackage{fancyhdr}
\pagestyle{fancy}
\fancyhead{}
\fancyfoot[RO,LE]{\scriptsize{\input{now}}}
\renewcommand{\headrulewidth}{0pt}

\input{polycode.lhs}

\usepackage{url}
\usepackage{longtable}
\usepackage{multirow}
\usepackage{comment}

%\def\TODO#1{\par\noindent{}TODO:~\ldots\emph{#1}\ldots\par}
\def\TODO#1{\noindent{}TODO:~\ldots\emph{#1}\ldots}

\def\source#1{}

\usepackage{amsmath}


% lhs2TeX
%{{{%
\usepackage[pdftex]{color}

\renewcommand\hsindent[1]{\quad}
\setlength{\mathindent}{2ex}
\DeclareMathAlphabet{\mathkw}{OT1}{cmss}{bx}{n}

\newcommand\cons[1]{\textcolor[rgb]{0,.5,0}{\mathsf{#1}}}
\newcommand\type[1]{\textcolor[rgb]{0,0,.75}{\mathsf{#1}}}
\newcommand\keyw[1]{\textcolor[rgb]{.75,.4,0}{\mathkw{#1}}}
\newcommand\func[1]{\textcolor[rgb]{.75,0,0}{\mathsf{#1}}}
\newcommand\name[1]{\textcolor[rgb]{.75,0,.75}{\mathtt{#1}}}
\newcommand\commentstyle[1]{\textcolor[rgb]{0,.6,0.75}{#1}}

\newcommand\Prime{\ensuremath{'}}
\newcommand\PPrime{\ensuremath{''}}

\newcommand\prefix[1]{#1\!}
\newcommand\postfix[1]{\!#1}

\newcommand\infix[1]{\mathbin{#1}}
\newcommand\infixL[1]{\infix{#1}\!}
\newcommand\infixM[1]{\!\infix{#1}\!}
\newcommand\infixR[1]{\!\infix{#1}}
%}}}%




\includeonly{semantics,agda.lagda}
\begin{document}

\begin{titlepage}
\begin{center}
\vspace*{1in}
{\LARGE Compiling Concurrency Correctly}
\par
{\Large Verifying Software Transactional Memory}
\par
\vspace{1.5in}
{\large Liyang HU}
\par
\vfill
A Thesis submitted for the degree of Doctor of Philosophy
\par
\vspace{0.5in}
School of Computer Science
\par
\vspace{0.5in}
University of Nottingham
\par
\vspace{0.5in}
June 2010
\end{center}
\end{titlepage}

\pagenumbering{roman}

\begin{abstract}%{{{%
Concurrent programming is notoriously difficult, but with multi-core
processors becoming the norm, it is now a reality that every programmer must
face. Concurrency has traditionally been managed using low-level
mutual exclusion \emph{locks}, which are error-prone and do not naturally
support the compositional style of programming that is becoming
indispensable for today's large-scale software projects.

A novel, high-level approach that has emerged in recent years is that of
\emph{software transactional memory} (STM), which avoids the need for
explicit locking, instead presenting the programmer with a declarative
approach to concurrency. However, its implementation is much more complex
and subtle, and ensuring its correctness places significant demands on the
compiler writer.

This thesis considers the problem of formally verifying a compiler for STM.
Based on a minimal language, we first explore various STM design choices,
along with the issue of compiler correctness via the use of automated
testing tools. Then we outline a new approach to concurrent compiler
correctness using the notion of bisimulation, implemented using the Agda
theorem prover. Finally, we apply this approach to our minimal language to
give the first formally verified compiler for software transactional memory.
\end{abstract}%}}}%

\chapter*{Acknowledgements}
...

\tableofcontents
%\listoffigures

\pagenumbering{arabic}

\include{introduction.lhs}
\include{stm.lhs}
%include local.fmt

\def\prod{\mathrel{::=}}
\def\altn{\mathrel{\mid}}
\def\NT#1{\mathsf{#1}}
\def\Nat{\mathbb{N}}
\def\Expression{\NT{Expression}}

\chapter{Semantics for Compiler Correctness}

%\begin{itemize}
%\item denotational, small-step for $(+,N)$
%\item equivalence proof
%\item rule induction (coinduction?)
%\item machine semantics for $\{PUSH m, ADD\} x N$
%\item statement of compiler correctness (style? direct vs CPS)
%\item compiler correctness as running example: proof
%\end{itemize}

In the context of computer science, semantics is the study of the meaning of
programming languages. Having a mathematically rigorous definition of the
language allows us to reason about (programs written in) the language
precisely and without ambiguity. In this chapter, we will take an elementary
look at different ways of giving meaning to a language, and various
techniques for proving properties about programs. After this prelude, we
reach the nub of this chapter, where we shall define a simple compiler from
an expression language to a stack machine, and explore what it means to say
that the compiler is correct.

\section{Semantics}%{{{%

\subsection{Natural Numbers and Addition}%{{{%

To unambiguously reason about what any given program means, we need to give
a mathematically rigorous definition of the language in which it is
expressed. To this end, let us consider the elementary language of natural
numbers and
addition~\cite{hutton04-exceptions,hutton06-calculating,hutton07-interruptions}.
\begin{align*}
	\Expression
		& \prod \Nat \eqTag{Exp-$\mathbb{N}$} \\
		& \altn \Expression \oplus \Expression \eqTag{Exp-$\oplus$}
\end{align*}
That is, an $\Expression$ is either simply a natural number, or a pair of
$\Expression$s, punctuated with the $\oplus$ symbol to represent the
operation of addition. We will adhere to a naming convention of $m, n \in
\Nat$ and $a, b, e \in \Expression$.

Although seemingly simplistic, this language has sufficient structure to
illustrate the essential aspects of computation, namely that of sequencing
computations and combining their results, as we shall expand upon later in
sections \ref{sec:small-step} and \ref{sec:monoid}.

%}}}%

\subsection{Denotational Semantics}%{{{%

% semantic brackets
\def\sb[#1]{[\![#1]\!]}

Denotation semantics attempts to give an interpretation of the source
language in some suitable existing formalism that we already understand.
More specifically, the denotation of a program is a representation of what
the program means in the vocabulary of the chosen formalism, which could be
the language of sets and functions, the $\lambda$-calculus, or perhaps one
of the many process calculi. Thus, to formally give a denotational semantics
for a language is to define a mapping from the source language into some
underlying semantic domain.

For example, we can give the following semantics for our earlier
$\Expression$ language, denoted as a natural number:
\begin{align*}
	\sb[\anonymous] &: \Expression \rightarrow \Nat \\
	\sb[ m ] &= m \eqTag{denote-val} \\
	\sb[ a \oplus b ] &= \sb[ a ] + \sb[ b ] \eqTag{denote-plus}
\end{align*}
Here, a numeric $\Expression$ is interpreted as just the number itself. The
denotation of the $\oplus$ operator is the familiar $+$ on natural numbers;
alternatively, we could say that $a \oplus b$ is denoted by the sum of the
denotations of its sub-expressions $a$ and $b$. The expression $\sb[(1
\oplus 2) \oplus (4 \oplus 8)]$ say, has the denotation $15$ by repeatedly
applying the above definition:
\begin{align*}
	\sb[(1 \oplus 2) \oplus (4 \oplus 8)]
		&= \sb[1 \oplus 2] + \sb[4 \oplus 8] \\
		&= (\sb[1] + \sb[2]) + (\sb[4] + \sb[8]) \\
		&= (1 + 2) + (4 + 8) = 15
\end{align*}

%}}}%

\subsection{Big-Step Operational Semantics}%{{{%

The notion of big-step operational semantics is concerned with the overall
result of a computation. Formally, we define a relation ${\Downarrow}
: \Expression \times \Nat$ between $\Expression$s and their final values,
given below in a natural deduction style:
\begin{gather*}
\inferrule*{ }
	{m \Downarrow m} \eqTag{big-val} \\[2ex]
\inferrule*{a \Downarrow m \quad b \Downarrow n}%
	{a \oplus b \Downarrow m + n} \eqTag{big-plus}
\end{gather*}
The first \eqName{big-val} rule says that a simple numeric $\Expression$
evaluates to the number itself. The second \eqName{big-plus} rule states
that, if $a$ evaluates to $m$ and $b$ evaluates to $n$, then $a \oplus b$
evaluates to the sum $m + n$. Thus according to this semantics, $(1 \oplus
2) \oplus (4 \oplus 8) \Downarrow 15$ by the following derivation:
\begin{gather*}
\inferrule* [left=big-plus]
{
	\inferrule* [Left=big-plus]
	{
		\inferrule* [Left=big-val]
			{ }{1 \Downarrow 1}
		\and
		\inferrule*
			{ }{2 \Downarrow 2}
	}
	{1 \oplus 2 \Downarrow 3}
	\and
	\inferrule*
	{
		\inferrule*
			{ }{4 \Downarrow 4}
		\and
		\inferrule*
			{ }{8 \Downarrow 8}
	}
	{4 \oplus 8 \Downarrow 12}
}
{(1 \oplus 2) \oplus (4 \oplus 8) \Downarrow 15}
\end{gather*}

\noindent One advantage of a relational operational semantics is that the
behaviour can be non-deterministic, in the sense that each expression could
potentially evaluate to multiple distinct values. In contrast,
a denotational semantics deals with non-determinism in the source language
by mapping it to a potentially different notion of non-determinism in the
underlying formalism. Should we require our expression language to be
non-deterministic, we would need to switch the denotational domain of the
previous semantics to the power set of natural numbers, rather than just the
set of natural numbers.

%}}}%

\subsection{Small-Step Operational Semantics}\label{sec:small-step}%{{{%

Small-step semantics on the other hand is concerned with how a computation
proceeds as a sequence of steps. Both big-step and small-step semantics are
`operational' in the sense that the meaning of a program is understood
through how it operates to arrive at the result. However, in this case each
reduction step is made explicit, which is particularly apt when we wish to
consider computations that produce side-effects. Again we formally define
a relation ${\mapsto} : \Expression \times \Expression$, but between pairs
of $\Expression$s in this instance:
\begin{gather*}
\inferrule*{ }%
	{m \oplus n \mapsto m + n} \eqTag{small-plus} \\[2ex]
\inferrule*{b \mapsto b'}%
	{m \oplus b \mapsto m \oplus b'} \eqTag{small-right} \\[2ex]
\inferrule*{a \mapsto a'}%
	{a \oplus b \mapsto a' \oplus b} \eqTag{small-left}
\end{gather*}
The first rule \eqName{small-plus} deals with the case where the expressions
on both sides of $\oplus$ are numerals: in a single step, it reduces to the
sum $m + n$. The second \eqName{small-right} rule applies when the left
argument of $\oplus$ is a numeral, in which case the right argument makes
a single reduction, while \eqName{small-left} reduces the left argument of
$\oplus$ whenever possible. There is no rule corresponding to a lone numeric
$\Expression$ as no further reductions are possible in this case.

As each ${\mapsto}$ corresponds to only a single computational step, it will
often be more convenient to refer to it via its reflexive, transitive
closure, defined as follows:
\begin{gather*}
\inferrule*[right=small-nil]
{ }{a \mapsto^\star a}
\qquad\qquad
\inferrule*[right=small-cons]
{a \mapsto a' \and a' \mapsto^\star b}
{a \mapsto^\star b}
\end{gather*}
Thus, the full reduction sequence of $(1 \oplus 2)
\oplus (4 \oplus 8) \mapsto^\star 15$ would begin at the $1 \oplus 2$
sub-expression,
\begin{gather*}
\inferrule* [left=small-left]
{
	\inferrule* [Left=small-plus]
	{ }{1 \oplus 2 \mapsto 3}
}
{(1 \oplus 2) \oplus (4 \oplus 8) \mapsto 3 \oplus (4 \oplus 8)}
\end{gather*}
followed by $4 \oplus 8$,
\begin{gather*}
\inferrule* [left=small-right]
{
	\inferrule* [Left=small-plus]
	{ }{4 \oplus 8 \mapsto 12}
}
{3 \oplus (4 \oplus 8) \mapsto 3 \oplus 12}
\end{gather*}
before delivering the final sum:
\begin{gather*}
\inferrule* [left=small-plus]
{ }{3 \oplus 12 \mapsto 15}
\end{gather*}

\noindent It would be perfectly reasonable to give a right-to-left, or even
a non-deterministic interleaved reduction strategy for our $\Expression$
language. However, we enforce a left-to-right order to coincide with the
definition of $\bind$ for the $\mathsf{State}\;\alpha$ monad of \S\ref{?},
which we motivate in the following section.

%}}}%

\subsection{Monoids as Degenerate Monads}\label{sec:monoid}%{{{%

In the previous chapter, we briefly explored the use of monads as
a mechanism for modelling sequential computations. In the degenerate case
where the result type of the computations form a monoid with an operation
$\cdot$, such computations themselves naturally form a monoid too.
Concretely, suppose we are working in some monad $\mathsf{M}$ computing
values of type $\Nat$, where $(\Nat,\;+,\;0)$ is a monoid. With the
following definition of $\ast$,
\[
	a \ast b \quad=\quad
		a \bind \lambda m \rightarrow
		b \bind \lambda n \rightarrow
		\textit{return}\;(m + n)
\]
such computations themselves form the monoid
$(\mathsf{M}\;\Nat,\;\ast,\;\textit{return}\;0)$. As we have done in
\S\ref{monad-laws}, we can again use equational reasoning to confirm that
the relevant left-/right-identity and associativity properties hold.
Therefore, it is not entirely unreasonable to view monoids as a degenerate
form of monads.

As we only consider values of natural numbers, rather than dealing with
computations of type $\mathsf{M}\;\Nat$, we may as well deal directly with
the underlying $(\Nat,\;+)$ monoid, mirrored in the syntax of the
$\Expression$ language. This simplification allows us to avoid the
orthogonal issues of binding and substitution. By enforcing a left-to-right
evaluation order for $\oplus$ to mirror that of $\bind$, we maintain the key
monadic aspect of sequencing computations.

%}}}%

\TODO{Are the proofs necessary? They're rather elementary.}
%{{}}%
%{{}}%
%{{{%
\begin{verbatim}
  (a * b) * c
={ defn of * }
  (a >>= \ l -> b >>= \ m -> return (l . m)) >>= \ lm ->
  c >>= \ n ->
  return (lm . n)
={ associativity of >>=, twice }
  a >>= \ l -> b >>= \ m -> (return (l . m) >>= \ lm ->
  c >>= \ n ->
  return (lm . n))
={ substitute lm by left-id of >>= }
  a >>= \ l ->
  b >>= \ m ->
  c >>= \ n ->
  return ((l . m) . n)
={ associativity of . }
  a >>= \ l ->
  b >>= \ m ->
  c >>= \ n ->
  return (l . (m . n))
={ factor out (m . n)  }
  a >>= \ l ->
  b >>= \ m -> c >>= \ n -> (return (m . n) >>= \ mn ->
  return (l . mn))
={ associativity of >>=, twice }
  a >>= \ l ->
  (b >>= \ m -> c >>= \ n -> return (m . n)) >>= \ mn ->
  return (l . mn)
={ defn of * }
  a * (b * c)
\end{verbatim}
%}}}%

%}}}%

\section{Equivalence Proofs and Techniques}%{{{%

We can now proceed to show various properties of the $\Expression$ language
in a rigorous manner, now that we have provided precise definitions for the
semantics of the language. One obvious questions arises, on the matter of
whether the semantics we have given in the previous section---denotational,
big-step and small-step---agree in some manner. This section reviews the
main techniques for proving such properties.

\subsection{Syntax and Rule Induction}%{{{%

\def\subExp{\sqsubset}

The main tool at our disposal is that of well-founded induction, which we
can apply to any well-founded structure. For example, we can show that the
syntax of the $\Expression$ language satisfies the condition of
well-foundedness when paired with the following sub-expression ordering:
\[
	a \subExp a \oplus b \qquad b \subExp a \oplus b \eqTag{Exp-$\subExp$}
\]
The partial order given by the transitive closure of $\subExp$ is
well-founded, since any $\subExp$-descending chain of expressions must
eventually end in a numeral at the leaves of the expression tree. This
particular ordering arises naturally from the inductive definition of
$\Expression$: the inductive case \eqName{Exp-$\oplus$} allows us to build
a larger expression $a \oplus b$ given two existing expressions $a$ and $b$,
while the base case \eqName{Exp-$\mathbb{N}$} constructs primitive
expressions out of any natural number. In this particular case, to give
a proof that some property $P(e)$ holds for all $e \in \Expression$, it
suffices by the well-founded induction principle to show instead that:
\[
	\forall b \in \Expression.\;
		(\forall a \in \Expression.\; a \subExp b \rightarrow P(a))
		\rightarrow P(b)
\]
More explicitly, we are provided with the hypothesis that $P(a)$ already
holds for all sub-expressions $a \subExp b$ when proving $P(b)$; in those
cases when $b$ has no sub-expressions, we must show that $P(b)$ holds
directly.

The application of well-founded induction to the structure of an inductive
definition is called \emph{structural induction}: to prove that a property
$P(x)$ holds for all members $x$ of an inductively defined structure $X$, it
suffices to initially show that $P(x)$ holds in all the base cases in the
definition of $X$, and that $P(x)$ holds in the inductive cases assuming
that $P(x')$ holds for any immediate substructure $x'$ of $x$.

Our earlier reduction rules ${\Downarrow}$ along with ${\mapsto}$ and its
transitive closure ${\mapsto^\star}$ are similarly inductively defined, and
therefore admits the same notion of structural induction. These instances
will be referred to as \emph{rule induction}.

%}}}%

\subsection{Proofs of Semantic Equivalence}%{{{%

We shall illustrate the above technique with some examples. Given that we
have defined 

\begin{theorem}
Denotational semantics and big-step operational semantics coincide:
\[
	\forall e \in \Expression,\ m \in \Nat.\quad
		\sb[e] \equiv m \leftrightarrow e \Downarrow m
\]
\end{theorem}

\begin{proof}
We consider each direction of the $\leftrightarrow$ biconditional
separately. To show $\sb[e] \equiv m \rightarrow e \Downarrow m$, we could
proceed by induction on the structure of the definition of the
$\sb[\anonymous]$ function, which happens to be structurally recursive on
its argument. Therefore we may equivalently proceed by structural induction
on $e$, giving us two cases to consider:
\begin{description}
\item[Case $e \equiv n$:]%{{}}%
\item[Case $e \equiv a \oplus b$:]%{{{%
Substituting $e$ as before, we need to show that:
\[
	\sb[a \oplus b] \equiv m \rightarrow a \oplus b \Downarrow m
\]
Applying \eqName{denote-plus} once to the hypothesis, we obtain that $\sb[a]
+ \sb[b] \equiv m$. Substituting for $m$, the conclusion becomes $a \oplus
b \Downarrow \sb[a] + \sb[b]$. Instantiate the induction hypothesis twice
with the trivial equalities $\sb[a] \equiv \sb[a]$ and $\sb[b] \equiv
\sb[b]$ to yield proofs of $a \Downarrow \sb[a]$ and $b \Downarrow \sb[b]$,
which are precisely the two antecedents required by \eqName{big-plus} to
obtain $a \oplus b \Downarrow \sb[a] + \sb [b]$.
%}}}%
\end{description}

\noindent The second half of the proof requires us to show that $\sb[e]
\equiv m \leftarrow e \Downarrow m$. We may proceed by structural induction
directly on our assumed hypothesis of $e \Downarrow m$, which must match
either \eqName{big-val} or \eqName{big-plus} in the definition of
$\Downarrow$:
\begin{description}
\item[Rule \eqName{big-val}:]%{{}}%
\item[Rule \eqName{big-plus}:]%{{{%
Again by matching $e \Downarrow m$ with the consequent of \eqName{big-plus},
there exists $a$, $b$, $n_a$ and $n_b$ where $e \equiv a \oplus b$ and $m
\equiv n_a + n_b$, such that $a \Downarrow n_a$ and $b \Downarrow n_b$.
Substituting for $e$ and $m$, the conclusion becomes $\sb[a \oplus b] \equiv
n_a + n_b$, which reduces to:
\[
	\sb[a] + \sb[b] \equiv n_a + n_b
\]
by applying \eqName{denote-plus} once. Instantiating the induction
hypothesis twice with $a \Downarrow n_a$ and $b \Downarrow n_b$ yields the
equalities $\sb[a] \equiv n_a$ and $\sb[b] \equiv n_b$ respectively, which
allows us to rewrite the conclusion as $\sb[a] + \sb[b] \equiv \sb[a]
+ \sb[b]$ by substituting $n_a$ and $n_b$. The desired result is now
trivially true by reflexivity of $\equiv$.
%}}}%
\end{description}

\noindent Thus we have shown both directions of the theorem.
\end{proof}

% premise, antecedent / consequent
% hypothesis / conclusion




\begin{theorem}
Big-step and small-step operational semantics coincide. That is,
\[
	\forall e \in \Expression,\ m \in \Nat.\quad
		e \Downarrow m \leftrightarrow e \mapsto^\star m
\]
\end{theorem}
\begin{proof}
We shall consider each direction separately as before. To show the forward
implication, we proceed by rule induction on the assumed $e \Downarrow m$
hypothesis:
\begin{description}
\item[Rule \eqName{big-val}:]%{{}}%
\item[Rule \eqName{big-plus}:]%{{{%
There exists $a$, $b$, $n_a$ and $n_b$ where $e \equiv a \oplus b$ and $m
= n_a + n_b$, such that $a \Downarrow n_a$ and $b \Downarrow n_b$. After
substituting for $e$ and $m$, the desired conclusion becomes:
\[
	a \oplus b \mapsto^\star n_a + n_b
\]
Instantiating the induction hypothesis with $a \Downarrow n_a$ and $b
\Downarrow n_b$ gives us evidence of $a \mapsto^\star n_a$ and $b
\mapsto^\star n_b$ respectively. With the former, we can apply $\anonymous
\oplus b$ to each of the terms and \eqName{small-left} to obtain a proof of
$a \oplus b \mapsto^\star n_a \oplus b$, while with the latter, we obtain
$n_a \oplus b \mapsto^\star n_a \oplus n_b$ by applying $n_a \oplus
\anonymous$ and \eqName{small-right}.

By the transitivity of $\mapsto^\star$, these two small-step reduction
sequences combine to give $a \oplus b \mapsto^\star n_a \oplus n_b$, to
which we need only append an instance of \eqName{small-plus} to arrive at
the conclusion.
%}}}%
\end{description}

\noindent The converse implication $e \Downarrow m \leftarrow
e \mapsto^\star m$ additionally requires lemma \ref{lem:small-sound},
which states that $e \mapsto e' \rightarrow e' \Downarrow m \rightarrow
e \Downarrow m$; in other words, the reduct of a single step under the
small-step semantics evaluates under the big-step semantics to the same
value as the original expression. We proceed by induction over the
definition of $\mapsto^\star$, given $e \mapsto^\star m \rightarrow
e \Downarrow m$:
\begin{description}
\item[Rule \eqName{small-nil}:]%{{}}%
\item[Rule \eqName{small-cons}:]%{{}}%
\end{description}
Pending the proof of lemma \ref{lem:small-sound} below, we have thus shown
the equivalence of big- and small-step semantics for the $\Expression$
language.
\end{proof}

\begin{lemma}
\label{lem:small-sound}
A single small-step reduction preserves the value of expressions with
respect to the big-step semantics:
\[
	\forall e, e' \in \Expression,\ m \in \Nat.\quad
		e \mapsto e' \rightarrow
		e' \Downarrow m \rightarrow e \Downarrow m
\]
\end{lemma}
\begin{proof}
Assume the two premises $e \mapsto e'$ and $e' \Downarrow m$, and proceed by
induction on the structure of the first:
\begin{description}
\item[Rule \eqName{small-plus}:]%{{{%
There are $n_a$ and $n_b$ such that $e \equiv n_a \oplus n_b$ and $e' \equiv
n_a + n_b$. As $e'$ is a numeric expression, the only applicable rule for
$e' \Downarrow m$ is \eqName{big-val}, which implies $m \equiv n_a + n_b$.
Thus the desired conclusion of $e \Downarrow m$---after substituting for $e$
and $m$---may be satisfied as follows:
\begin{gather*}
\inferrule* [Left=big-plus]
{
	\inferrule* [Left=big-val]
	{ }{n_a \Downarrow n_a}
	\and
	\inferrule*
	{ }{n_b \Downarrow n_b}
}{n_a \oplus n_b \Downarrow n_a + n_b}
\end{gather*}
%}}}%
\item[Rule \eqName{small-right}:]%{{{%
There exists $n_a$, $b$ and $b'$ such that $b \mapsto b'$ with $e \equiv n_a
\oplus b$ and $e' \equiv n_a \oplus b'$. Substituting for $e'$, the second
assumption becomes $n_a \oplus b' \Downarrow m$, with \eqName{big-plus} as
the only matching rule. This implies the existence of the premises $n_a
\Downarrow n_a$ and $b' \Downarrow n_b$,
\begin{gather*}
\inferrule*
{
	\inferrule*
	{ }{n_a \Downarrow n_a}
	\and
	\inferrule*
	{
		\vdots
	}{b' \Downarrow n_b}
}{n_a \oplus b' \Downarrow n_a + n_b}
\end{gather*}
for some $n_b$ such that $m \equiv n_a + n_b$. Invoking the induction
hypothesis with $b \mapsto b'$ and the above derivation of $b' \Downarrow
n_b$, we obtain a proof of $b \Downarrow n_b$. The conclusion is satisfied
by the following derivation:
\begin{gather*}
\inferrule*
{
	\inferrule*
	{ }{n_a \Downarrow n_a}
	\and
	\inferrule* [Right=IH]
	{
		\vdots
	}{b \Downarrow n_b}
}{n_a \oplus b \Downarrow n_a + n_b}
\end{gather*}
%}}}%
\item[Rule \eqName{small-left}:]%{{{%
This case proceeds in a similar manner to the previous rule, but with $a$,
$a'$ and $b$ such that $a \mapsto a'$, where $e \equiv a \oplus b$ and $e'
\equiv a' \oplus b$. Substituting for $e$ and $e'$ in the second assumption
and inspecting its premises, we observe that $a' \Downarrow n_a$ and $b
\Downarrow n_b$ for some $n_a$ and $n_b$ where $m \equiv n_a + n_b$:
\begin{gather*}
\inferrule*
{
	\inferrule*
	{
		\vdots
	}{a' \Downarrow n_a}
	\and
	\inferrule*
	{
		\vdots
	}{b \Downarrow n_b}
}{a' \oplus b \Downarrow n_a + n_b}
\end{gather*}
Instantiating the induction hypothesis with $a \mapsto a'$ and $a'
\Downarrow n_a$ delivers evidence of $a \Downarrow n_a$. Reusing the second
premise of $b \Downarrow n_b$ verbatim, we can then derive the
conclusion of $a \oplus b \Downarrow n_a + n_b$:
\begin{gather*}
\inferrule*
{
	\inferrule* [Left=IH]
	{
		\vdots
	}{a \Downarrow n_a}
	\and
	\inferrule*
	{
		\vdots
	}{b \Downarrow n_b}
}{a \oplus b \Downarrow n_a + n_b}
\end{gather*}
%}}}%
\end{description}
This completes the proof of $e \mapsto e' \rightarrow e' \Downarrow
m \rightarrow e \Downarrow m$.
\end{proof}

%}}}%

%}}}%


\section{Compiler Correctness}%{{{%

\subsection{Stack Machines and Their Semantics}%{{}}%

\subsection{Compiler}%{{}}%

\subsection{Proof of High-Level and Virtual Machine Semantics}%{{}}%

%}}}%

% vim: ft=tex:


\include{testing.lhs}
\include{model.lhs}
\include{agda.lagda}
\include{nondet.lagda}
\include{concurrency.lagda}
\include{verified.lagda}
\chapter{Conclusion}

To conclude, these final pages will comprise an overview of this thesis and
an account of how it came to be, followed by a summary of its contributions
and some directions for further work.

\section{Retrospection}%{{{%

The quest for higher-level abstractions to manage the complexities of
concurrent programming has been an especially apt topic in recent years, due
to reasons outlined in the introductory chapter. With respect to software
transactional memory (Chapter \ref{ch:stm}), I was fortunate enough to be in
the right places at the right times to have attended two of Tim Harris's
talks on the topic: the first in Cambridge during my undergraduate years, on
the JVM-based implementation; and a second time at Imperial College during
my MSc course in the early part of 2005, on the composability of STM
Haskell.

My work for this thesis began in 2006 under the guidance of Graham Hutton,
with an initial goal of reasoning about concurrent programs, in particular
those written using STM. To this end, we opted for a simple formal language
based on Hutton's Razor, extended with a minimal set of transactional
primitives, described in Chapter \S\ref{ch:model}. While this
language---following the reference stop-the-world semantics given by Harris
et al.~\cite{harris05-composable}---had a simple implementation, it was not
immediately clear how STM Haskell dealt with conflicting transactions
internally, consequently drawing our attention towards the correctness of
the low-level concurrent implementation.

To better understand the implementation issues behind software transactional
memory, we began building a stack-based virtual machine and a compiler for
our minimal language, of which the final version is given in
\S\ref{sec:model-machine}. Using Haskell as a metalanguage, it was
a straightforward task to transcribe the syntax and semantics of our model
as an executable program. Combined with the use of QuickCheck and the
Haskell Program Coverage toolkit (Chapter \ref{ch:qc+hpc}), this allowed us
to take a `rapid prototyping' approach to the design of the virtual machine.
Most notably, we were able to clarify the appropriate conditions needed for
a transaction to commit successfully (\S\ref{sec:model-equality}), and to
realise that writing to an as-yet unread variable within a transaction does
not imply a dependency on the current state of the heap.

Of course, regardless of how many times we run QuickCheck on our STM model,
overwhelming evidence does not constitute a proof of compiler correctness
for our interleaved semantics and log-based implementation of transactional
memory. Due to the influence of numerous type-theorists at Nottingham, I had
become interested in dependently-typed programming (Chapter \ref{ch:agda})
and dually, the application of intuitionistic type theory as a framework for
conducting formal proofs. Therefore, the goal of a mechanised compiler
correctness proof for our model in a dependently-typed
language/proof-assistant seemed a natural choice.

Whereas compiler correctness theorems in a deterministic setting (Chapter
\ref{ch:semantics}) are concerned only with the final results, with the
introduction of non-determinism (of which concurrency is one form) we can no
longer afford to ignore \emph{how} results are computed, in addition to what
is being computed. Bisimilarity as a notion of behavioural equivalence is
a standard tool in the field of process calculi, and
Wand~\cite{wand95-parallel,wand95-denotational,gladstein96-concurrent} et
al.~were the first to use it to tackle concurrent compiler correctness over
a decade ago. Their work relied on giving denotational semantics to both
source and target languages in an underlying process calculus, and showing
that compilation preserved bisimilarity of denotations. In contrast, we
defined our language and virtual machine in an operational manner, and
sought a simpler and more direct approach.

Thus, the idea of the combined machine was born, detailed in Chapter
\ref{ch:nondet}. A key realisation was that certain kinds transitions
preserve bisimilarity, giving rise to the $\func{elide\text-\tau}$ lemma.
Having tested the waters with the non-deterministic Zap language, Chapter
\ref{ch:fork} then demonstrates that our approach can indeed scale to handle
concurrency, at least for that of the Fork language. As well as updating the
$\func{elide\text-\tau}$ lemma for explicit concurrency, we also showed that
combining bisimilar pairs of thread soups preserves bisimilarity.

Finally, Chapter \ref{ch:verified} brings our object language up to par with
our earlier STM model by introducing an $\cons{atomic}$ construct that
follows a stop-the-world semantics, in contrast with its virtual machine's
log-based implementation that allows concurrent transactions to interleave
each other. The final compiler correctness proof makes essential use of
a notion of equivalence between the heap of the high-level semantics and
that of the virtual machine, overlaid with read and write logs.

%}}}%

\section{Summary of Contributions}%{{{%

This thesis addresses the familiar question of compiler correctness in
a current context, with a particular emphasis on the implementation of
software transactional memory. We have identified a simplified subset of STM
Haskell that is amenable to formal reasoning, which has a stop-the-world
transactional semantics, together with a concurrent virtual machine for this
language, using the notion of transaction logs. A compiler linking this
simplified language to its virtual machine then allowed us to formulate
a concrete statement of compiler correctness. We were able to implement the
above semantics in a fairly direct manner using the high-level vocabulary
provided by Haskell, enabling us to empirically test compiler correctness
with the help of QuickCheck and HPC.

Working towards a formal proof of the above hypothesis, we stripped down to
a minimal language with trivial non-determinism, and moving to a labelled
transition system. The core idea of a combined machine and semantics then
allowed us to establish a direct bisimulation between this language and its
virtual machine. This technique was put into practice using the Agda proof
assistant, giving a machine-checked compiler correctness proof for
a language with a simple notion of non-determinism. We then extended the
above proof and our approach in the direction of the initially identified
subset of STM Haskell, in an incremental manner: first introducing explicit
concurrency in the form of a $\cons{fork}$ primitive, before finally
extending this language with an $\cons{atomic}$ construct. We believe this
to be the first machine-checked compiler correctness proof showing the
equivalence of the log-based approach and the stop-the-world semantics for
implementing transactions.

%}}}%

\section{Directions for Further Research}%{{{%

Our simplified model of STM Haskell focuses on the essence of implementing
transactions, and consequently omits many of the facilities expected of
a realistic language. Namely, the lack of primitive recursion or even name
binding limits the computational power of our model in a very tangible
sense. We could be tackle this using a lightweight approach, by borrowing
said facilities from the metalanguage and defining the high-level semantics
as a functional specification. For example, Gordon's
thesis~\cite{gordon92-fpio} presents such a specification of teletype IO for
a subset of Haskell in terms of a low-level metalanguage, while
Swierstra~\cite{swierstra08-funspec} advocates the use of Agda as the
metalanguage, due to its enforcement of totality.

Given the above as a basis, a machine-verified formalisation of the omitted
parts of the STM Haskell specification---in particular
$\func{retry}$/$\func{orElse}$ and the interaction with exceptions---becomes
a much more tractable proposition. Open questions include: How will these
additions affect the design of the corresponding virtual machine? Can we
maintain the simplicity of our combined machine approach? Is the outline of
our reasoning for transactions still valid in this richer language? Our
current virtual machine immediately retries failed transactions, rather than
waiting until some relevant transactional variable has changed. How can our
virtual machine more faithfully model the implementation of STM in GHC?

Going further, we could extend the set of side-effects that can be safely
rolled back by the transactional machinery. One widely asserted advantage of
STM Haskell over other STM implementations is that its type system restricts
transactions---aside from modifying $\type{TVar}$s---to pure computations,
guaranteeing that rollback is always possible. During my initial work on the
model of STM Haskell, the notion of running multiple nested transactions
concurrently arose quite naturally, when considering their r\^{o}le in the
implementation of $\func{retry}$/$\func{orElse}$. While the $\func{forkIO}$
primitive is considered impure, forking a nested transaction need not be, as
its side-effects can only escape as part of that of its enclosing
transaction. It could be interesting to flesh out the precise behaviour of
such
a $\func{forkSTM}\;\keyw{::}\;\type{STM\;()\;\rightarrow\;STM\;ThreadId}$
primitive, and to evaluate its utility for concurrent programming in the
real world.

%}}}%

%\section*{Closing Remarks}

%Thanks for making it this far. It felt like I almost didn't.

% vim: ft=tex:



\bibliographystyle{alpha}
\bibliography{thesis}

\end{document}

