\chapter{Conclusion}

To conclude, these final pages will comprise an overview of this thesis and
an account of how it came to be, followed by a summary of its contributions
and some directions for further work.

\section{Retrospection}%{{{%

The quest for higher-level abstractions to manage the complexities of
concurrent programming has been an especially apt topic in recent years, due
to reasons outlined in the introductory chapter. With respect to software
transactional memory (Chapter \ref{ch:stm}), I was fortunate enough to be in
the right places at the right times to have attended two of Tim Harris's
talks on the topic: the first in Cambridge during my undergraduate years, on
the JVM-based implementation; and a second time at Imperial College during
my MSc course in the early part of 2005, on the composability of STM
Haskell.

My work for this thesis began in 2006 under the guidance of Graham Hutton,
with an initial goal of reasoning about concurrent programs, in particular
those written using STM. To this end, we opted for a simple formal language
based on Hutton's Razor, extended with a minimal set of transactional
primitives, described in Chapter \S\ref{ch:model}. While this
language---following the reference stop-the-world semantics given by Harris
et al.~\cite{harris05-composable}---had a simple implementation, it was not
immediately clear how STM Haskell dealt with conflicting transactions
internally, consequently drawing our attention towards the correctness of
the low-level concurrent implementation.

To better understand the implementation issues behind software transactional
memory, we began building a stack-based virtual machine and a compiler for
our minimal language, of which the final version is given in
\S\ref{sec:model-machine}. Using Haskell as a metalanguage, it was
a straightforward task to transcribe the syntax and semantics of our model
as an executable program. Combined with the use of QuickCheck and the
Haskell Program Coverage toolkit (Chapter \ref{ch:qc+hpc}), this allowed us
to take a `rapid prototyping' approach to the design of the virtual machine.
Most notably, we were able to clarify the appropriate conditions needed for
a transaction to commit successfully (\S\ref{sec:model-equality}), and to
realise that writing to an as-yet unread variable within a transaction does
not imply a dependency on the current state of the heap.

Of course, regardless of how many times we run QuickCheck on our STM model,
overwhelming evidence does not constitute a proof of compiler correctness
for our interleaved semantics and log-based implementation of transactional
memory. Due to the influence of numerous type-theorists at Nottingham, I had
become interested in dependently-typed programming (Chapter \ref{ch:agda})
and dually, the application of intuitionistic type theory as a framework for
conducting formal proofs. Therefore, the goal of a mechanised compiler
correctness proof for our model in a dependently-typed
language/proof-assistant seemed a natural choice.

Whereas compiler correctness theorems in a deterministic setting (Chapter
\ref{ch:semantics}) are concerned only with the final results, with the
introduction of non-determinism (of which concurrency is one form) we can no
longer afford to ignore \emph{how} results are computed, in addition to what
is being computed. Bisimilarity as a notion of behavioural equivalence is
a standard tool in the field of process calculi, and
Wand~\cite{wand95-parallel,wand95-denotational,gladstein96-concurrent} et
al.~were the first to use it to tackle concurrent compiler correctness over
a decade ago. Their work relied on giving denotational semantics to both
source and target languages in an underlying process calculus, and showing
that compilation preserved bisimilarity of denotations. In contrast, we
defined our language and virtual machine in an operational manner, and
sought a simpler and more direct approach.

Thus, the idea of the combined machine was born, detailed in Chapter
\ref{ch:nondet}. A key realisation was that certain kinds transitions
preserve bisimilarity, giving rise to the $\func{elide\text-\tau}$ lemma.
Having tested the waters with the non-deterministic Zap language, Chapter
\ref{ch:fork} then demonstrates that our approach can indeed scale to handle
concurrency, at least for that of the Fork language. As well as updating the
$\func{elide\text-\tau}$ lemma for explicit concurrency, we also showed that
combining bisimilar pairs of thread soups preserves bisimilarity.

Finally, Chapter \ref{ch:verified} brings our object language up to par with
our earlier STM model by introducing an $\cons{atomic}$ construct that
follows a stop-the-world semantics, in contrast with its virtual machine's
log-based implementation that allows concurrent transactions to interleave
each other. The final compiler correctness proof makes essential use of
a notion of equivalence between the heap of the high-level semantics and
that of the virtual machine, overlaid with read and write logs.

%}}}%

\section{Summary of Contributions}%{{{%

This thesis addresses the familiar question of compiler correctness in
a current context, with a particular emphasis on the implementation of
software transactional memory. We have identified a simplified subset of STM
Haskell that is amenable to formal reasoning, which has a stop-the-world
transactional semantics, together with a concurrent virtual machine for this
language, using the notion of transaction logs. A compiler linking this
simplified language to its virtual machine then allowed us to formulate
a concrete statement of compiler correctness. We were able to implement the
above semantics in a fairly direct manner using the high-level vocabulary
provided by Haskell, enabling us to empirically test compiler correctness
with the help of QuickCheck and HPC.

Working towards a formal proof of the above hypothesis, we stripped down to
a minimal language with trivial non-determinism, and moving to a labelled
transition system. The core idea of a combined machine and semantics then
allowed us to establish a direct bisimulation between this language and its
virtual machine. This technique was put into practice using the Agda proof
assistant, giving a machine-checked compiler correctness proof for
a language with a simple notion of non-determinism. We then extended the
above proof and our approach in the direction of the initially identified
subset of STM Haskell, in an incremental manner: first introducing explicit
concurrency in the form of a $\cons{fork}$ primitive, before finally
extending this language with an $\cons{atomic}$ construct. We believe this
to be the first machine-checked compiler correctness proof showing the
equivalence of the log-based approach and the stop-the-world semantics for
implementing transactions.

%}}}%

\section{Directions for Further Research}%{{{%

Our simplified model of STM Haskell focuses on the essence of implementing
transactions, and consequently omits many of the facilities expected of
a realistic language. Namely, the lack of primitive recursion or even name
binding limits the computational power of our model in a very tangible
sense. We could be tackle this using a lightweight approach, by borrowing
said facilities from the metalanguage and defining the high-level semantics
as a functional specification. For example, Gordon's
thesis~\cite{gordon92-fpio} presents such a specification of teletype IO for
a subset of Haskell in terms of a low-level metalanguage, while
Swierstra~\cite{swierstra08-funspec} advocates the use of Agda as the
metalanguage, due to its enforcement of totality.

Given the above as a basis, a machine-verified formalisation of the omitted
parts of the STM Haskell specification---in particular
$\func{retry}$/$\func{orElse}$ and the interaction with exceptions---becomes
a much more tractable proposition. Open questions include: How will these
additions affect the design of the corresponding virtual machine? Can we
maintain the simplicity of our combined machine approach? Is the outline of
our reasoning for transactions still valid in this richer language? Our
current virtual machine immediately retries failed transactions, rather than
waiting until some relevant transactional variable has changed. How can our
virtual machine more faithfully model the implementation of STM in GHC?

Going further, we could extend the set of side-effects that can be safely
rolled back by the transactional machinery. One widely asserted advantage of
STM Haskell over other STM implementations is that its type system restricts
transactions---aside from modifying $\type{TVar}$s---to pure computations,
guaranteeing that rollback is always possible. During my initial work on the
model of STM Haskell, the notion of running multiple nested transactions
concurrently arose quite naturally, when considering their r\^{o}le in the
implementation of $\func{retry}$/$\func{orElse}$. While the $\func{forkIO}$
primitive is considered impure, forking a nested transaction need not be, as
its side-effects can only escape as part of that of its enclosing
transaction. It could be interesting to flesh out the precise behaviour of
such
a $\func{forkSTM}\;\keyw{::}\;\type{STM\;()\;\rightarrow\;STM\;ThreadId}$
primitive, and to evaluate its utility for concurrent programming in the
real world.

%}}}%

%\section*{Closing Remarks}

%Thanks for making it this far. It felt like I almost didn't.

% vim: ft=tex:

